\documentclass[a4paper]{article}

\usepackage{fullpage}
\usepackage[utf8]{inputenc}
\usepackage{graphicx}
\usepackage{epsfig}
\usepackage{caption}
\usepackage{subcaption}
\usepackage{latexsym}
\usepackage{amssymb,amsmath,amsthm}
\usepackage{cancel}
\usepackage[table]{xcolor}
\usepackage{listings}
\usepackage{mdwlist}
%\usepackage{pdflscape}
%\usepackage{rotating}
%\usepackage{mathrsfs}
%\usepackage{natbib}
%\usepackage{verbatim}
\usepackage{algorithm}
\usepackage{algpseudocode}



%\usepackage{tikz}
%\usetikzlibrary{shapes,arrows,backgrounds,fit}

\usepackage{url}
% should be the last package
\usepackage{hyperref}
% fix jumping to capture instead of figure
\usepackage[all]{hypcap}


\graphicspath{{./figures/}}


\bibliographystyle{plain}

\newtheorem{thm}{Theorem}[section]
\newtheorem{lem}[thm]{Lemma}
\newtheorem{defin}[thm]{Definition}
\newtheorem{cor}[thm]{Corollary}
\newtheorem{cla}[thm]{Claim}



\lstset{ %
  language=Matlab,                % the language of the code
  basicstyle=\tt\scriptsize,         % the size of the fonts that are used for the code
  numbers=left,                   % where to put the line-numbers
  xleftmargin=1cm,
  stepnumber=1,                   % the step between two line-numbers. If it's 1, each line 
                                  % will be numbered
  numbersep=5pt,                  % how far the line-numbers are from the code
%  backgroundcolor=\color{white},      % choose the background color. You must add \usepackage{color}
%  showspaces=false,               % show spaces adding particular underscores
%  showstringspaces=false,         % underline spaces within strings
%  showtabs=false,                 % show tabs within strings adding particular underscores
%  frame=single,                   % adds a frame around the code
%  rulecolor=\color{black},        % if not set, the frame-color may be changed on line-breaks within not-black text (e.g. commens (green here))
%  tabsize=2,                      % sets default tabsize to 2 spaces
%  captionpos=b,                   % sets the caption-position to bottom
%  breaklines=true,                % sets automatic line breaking
%  breakatwhitespace=false,        % sets if automatic breaks should only happen at whitespace
%  title=\lstname,                   % show the filename of files included with \lstinputlisting;
%                                  % also try caption instead of title
%  keywordstyle=\color{blue},          % keyword style
%  commentstyle=\color{dkgreen},       % comment style
%  stringstyle=\color{mauve},         % string literal style
%  escapeinside={\%*}{*)},            % if you want to add a comment within your code
%  morekeywords={*,...}               % if you want to add more keywords to the set
}


\newcommand{\dm}{\text{dm}^3}
\newcommand{\OtravelTo}{\text{O\_travelTo}}
\newcommand{\OtravelFrom}{\text{O\_travelFrom}}
\newcommand{\Vvolume}{\text{V\_volume}}

\author{Stefan Hepp, e0026640}

\title{Modeling and Solving Constrained Optimization Problems VU\\
  {\normalsize RMC Exercise}}
\date{\today}

\begin{document}

\maketitle

\section{Introduction}

This report describes the modeling and implementation of the Ready-Mixed Concrete (RMC) problem exercise using Gecode.

\section{Instance Specification and Preprocessing}

The problem instances are described using XML files. They contain lists of orders, vehicles and stations. The output
nodes in the XML files are ignored.

The provided data files contain many entries that are invalid, as well as problems that are not satisfiable. Vehicles
that have neither a normal volume nor a maximum volume set are ignored, as well as stations without names. For each
order at least one vehicle must exist which can fulfill this order, otherwise the problem does not have a solution. 

In order to limit the model to integer variables, volumes of concrete are converted into integer units of $\dm$. Time
spans such as the time to load vehicles are represented in minutes, while points in time such as the start time of an
order are represented in minutes since the earliest time stamp in the problem specification. 
Discharge rates of the vehicles are expressed as integer values in $\dm$ per minutes.

While multiple orders for the same construction yard may exist but since travel
times are specified per order, not per yard, the problem was slightly simplified by assuming every order is for a
different construction yard, making construction yards and orders equivalent.
In the following, we thus only consider orders. If multiple orders for the same construction yard may exist, the 
constraint that allows only one vehicle to be unloaded per order must be modified to allow only one vehicle to be
unloaded per yard, by using the yard identifier instead of the order identifier to check for overlaps.

The travel times between stations and orders are converted from a variable length list per order to a single dense
matrix $\OtravelTo$ containing the 
travel times from stations to orders and to a matrix $\OtravelFrom$ containing the travel times from orders to yards.
If no value for a pair of order and station is given in the problem specification, it is assumed that it is not possible
to travel that route and the corresponding entries in the matrices are set to a very large value to avoid solutions that
contain such a route.

The amount of concrete that a vehicle can deliver may depend on the order. The volumes of all vehicles are precalculated
and stored in a matrix $\Vvolume$ indexed by order and vehicle.

\medskip
In addition to the provided data files, a small example with just two orders, three vehicles and three stations was
created to test and experiment with the model.

\section{Representation of Deliveries}

The expected output of the RMC problem is a schedule for all vehicles which order has to be fulfilled from which
station at which point in time. There are at least three ways to represent this schedule. 

\begin{itemize}
\item A single list of deliveries
 might contain a vehicle index, a station index and an order index per delivery.

\item A list of deliveries for each order defines which vehicles serve this order.

\item A list of deliveries for each vehicle defines which orders are served by this vehicle.
\end{itemize}

The issue with all three representations is that the number of deliveries is not fixed, and it is not possible to have
the size of the delivery lists depend on decision variables.\footnote{It might be possible to store deliveries using
sets, but this has not been explored for this model.}
Instead, an upper bound on the number of deliveries is calculated, and an array of boolean variables defines per
delivery if it is used or not.

Since the first option uses only one single list, a tighter bound on the length of the list and thus on the number of
variables might be found. However, the other ways of modeling the problem simplify some ordering constraints, therefore
the first option was not further explored.

For this exercise a lists of deliveries for each vehicle were used. The advantage over lists per order is that it is
simpler to determine the travel times between stations and yards per vehicle, as the next trip of a vehicle can
be easily determined. The disadvantage is that the lag time between deliveries for a given order is much more difficult
to determine, but it was assumed that this is slightly less of an issue than determining the travel times for vehicles.


\section{Modeling and Constaints}

For each vehicle a list of deliveries containing different values per delivery exists. Those delivery lists have been implemented
as one matrix over vehicles and deliveries per value to represent, linearized into IntVarArrays.

\newcommand{\DOrder}{\text{D\_Order}}
\newcommand{\DStation}{\text{D\_Station}}
\newcommand{\DtLoad}{\text{D\_tLoad}}
\newcommand{\DtUnload}{\text{D\_tUnload}}
\newcommand{\maxD}{\text{maxD}}
\newcommand{\VDeliveries}{\text{V\_Deliveries}}
\newcommand{\ODeliveries}{\text{O\_Deliveries}}
\newcommand{\DUsed}{\text{D\_Used}}
\newcommand{\DdTtravelTo}{\text{D\_dTTravelTo}}
\newcommand{\DdTtravelFrom}{\text{D\_dTTravelFrom}}

\begin{itemize}
\item $\DOrder_{vd}$: The order vehicle $v$ is serving with delivery $d$.
\item $\DStation_{vd}$: The station vehicle $v$ is loaded at for delivery $d$.
\item $\DtLoad_{vd}$: The time when loading vehicle $v$ for delivery $d$ starts.
\item $\DtUnload_{vd}$: The time when unloading vehicle $v$ for delivery $d$ starts.
\end{itemize}

The size of each such matrix is $|V| \times |O| \cdot \maxD$, where $V$ is the list of available vehicles, $O$ the list of
orders, and $\maxD$ is an upper bound of deliveries per order per vehicle, and is currently derived as the number of
deliveries the smallest vehicle needs to fullfil the largest order on its own.

The number of deliveries per vehicle is stored as an array $\VDeliveries$ of size $|V|$. A boolean array $\DUsed$
defines for each delivery $vd$ if it contains valid values. To break symmetries, always the first $\VDeliveries_v$
deliveries are used, i.e.,
\[
  \DUsed_{vd} = (d < \VDeliveries_v)
\]

All unused deliveries are set to a fixed value:
\[
\begin{array}[b]{l}
  \DUsed_{vd} \vee \DOrder_{vd} = 0  \\
  \DUsed_{vd} \vee \DStation_{vd} = 0 \\
  \DUsed_{vd} \vee \DtLoad_{vd} = 0 \\
  \DUsed_{vd} \vee \DtUnload_{vd} = 0
\end{array}
\]

\medskip

Based on those structures we can now define various helper variables that will allow us to define constraints and the
cost function.

The travel time from a station to a yard $\DdTtravelTo$ for delivery $vd$ is defined as
\[
\begin{array}[b]{l}
(\DdTtravelTo_{vd} = \OtravelTo[\DOrder_{vd}, \DStation_{vd}] \wedge \DUsed_{vd}) \vee \\
(\DdTtravelTo_{vd} = 0 \wedge \neg \DUsed_{vd})
\end{array}
\]

The travel time from the yard back to the next station $\DdTtravelFrom$ of a delivery is defined to be 0 for the last
delivery $d = |O| \cdot \maxD - 1$ and for all other deliveries as
\[
\begin{array}[b]{l}
(\DdTtravelFrom_{v,d} = \OtravelFrom[\DOrder_{v,d+1}, \DStation_{v,d}] \wedge \DUsed_{v,d+1}) \vee \\
(\DdTtravelFrom_{v,d} = 0 \wedge \neg \DUsed_{v,d+1})
\end{array}
\]







\section{Improving the Model}



\section{Evaluation of the Model}


In hindsight, a model using unsorted sets for deliveries or a single list of deliveries (sorted by some value encoding
both time stamps and orders for symmetry breaking) that entails more complex constraints but fewer variables might have
been a better choice.


%%%%%%%%%%%%%%%%%%%%%%%%%%%%%%%%%%%%%%%%%%%%%%%%%%%%%%%%%%%%%%%%%%%%%%%%%%%%%%%
%\clearpage
%\nocite{*}
%\bibliography{references}


\end{document}

